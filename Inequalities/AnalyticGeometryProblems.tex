\section{Problems}
\begin{namedframe}{Analytic geometry problem 1}

	\begin{exampleblock}{Problem}
		If the line $2x - 3y - 6 = 0$ is reflected in the line $y = -x$, find the equation of the image line.
	\end{exampleblock}
	\pause
	\begin{block}{Solution}
		First, we find two points on the original line.
		\pause
		It's easiest to find the intercepts, since we have formulas for those.
		\sep
		The line given has the intercepts $(3, 0)$ and $(0, -2)$.
		\sep
		When reflecting in $y = -x$, we swap $x$ and $y$, and change the sign.
		\sep
		So, we get:
		\[(3, 0) \to (0, -3)\]
		\[(0, -2) \to (2, 0)\]
	\end{block}
\end{namedframe}
\begin{namedframe}{Analytic geometry problem 1 solution continued}
	\begin{block}{Solution}
		Next, we find an equation for a line which goes through $(0, -3)$ and $(2, 0)$.
		\sep
		\begin{align*}
			\uncover<+->{\frac{x}{a} + \frac{y}{b} &= 1\\}
			\uncover<+->{\frac{x}{2} + \frac{y}{-3} &= 1\\}
			\uncover<+->{3x - 2y - 6 &= 0}
		\end{align*}
		\uncover<+->{And that is our line.}
	\end{block}
\end{namedframe}
\begin{namedframe}{Analytic geometry problem 2}
	\begin{exampleblock}{Problem}
		If $A(3,5)$ and $B(11,11)$ are fixed points, find the point(s) $P$ on the $x$-axis such that the area of the triangle $ABP$ equals $30$.
	\end{exampleblock}
	\pause
	\begin{block}{Solution}
		Let $P = (p, 0)$.

		Then, we use the formula for area:
		\begin{align*}
			\uncover<+->{A &= \frac{1}{2} \abs{x_1y_2 + x_2y_3 + x_3y_1 - x_2y_1 - x_3y_2 - x_1y_3}\\}
			\uncover<+->{A &= \frac{1}{2} \abs{33 + 0 + 5p - 55 - 11p - 0}\\}
			\uncover<+->{\abs{-22 - 6p} &= 60}
		\end{align*}
	\end{block}
\end{namedframe}
\begin{namedframe}{Analytic geometry problem 2 solution continued}
	\begin{block}{Solution}
		\begin{align*}
			\uncover<+->{\abs{-22 - 6p} &= 60}\\
			\uncover<+->{p &= \frac{19}{3}, -\frac{41}{3}}
		\end{align*}
		\uncover<+->{So the points are $\left( \frac{19}{3}, 0 \right)$ and $\left( -\frac{41}{3}, 0 \right)$.}
	\end{block}
\end{namedframe}
\begin{namedframe}{Analytic geometry problem 3}
	\begin{exampleblock}{Problem}
		Given the circles $x^2 + y^2 = 4$ and $x^2 + y^2 - 6x + 2 = 0$, find the length of their common chord.
	\end{exampleblock}
	\pause
	\begin{block}{Solution}
		It is helpful to rewrite the second circle's equation as $(x - 3)^2  + y^2 = 7$ because this shows us more easily that the second circle is centred at $(3, 0)$.
		\pause
		Since the line joining the centres is horizontal, the common chord is vertical.
		\pause
		We can solve this easily:
		\begin{align*}
			x^2 + y^2 - 4 &= x^2 + y^2 - 6x + 2\\
			x &= 1
		\end{align*}
	\end{block}
\end{namedframe}
\begin{namedframe}{Analytic geometry problem 3 solution continued}
	\begin{block}{Solution}
		\[x = 1\]
		If we substitute this back into the equation for either of our circle, we find that the chord intersects the circles at $(1, \pm\sqrt{3})$.
		\sep
		So, the length of the entire chord is \pause $2\sqrt{3}$.
	\end{block}
\end{namedframe}
\begin{namedframe}{Analytic geometry problem 4}
	\begin{exampleblock}{Problem}
		A line has slope $-2$ and is a distance of $2$ units from the origin.
		What is the area of the triangle formed by this line and the axes?
	\end{exampleblock}
	\pause
	\begin{block}{Solution}
		Let the $x$-intercept be $k$.
		\sep
		What does this make the $y$-intercept? \pause $2k$.
		\sep
		So the equation of the line can be written as:
		\[2x + y - 2k = 0\]
	\end{block}
\end{namedframe}
\begin{namedframe}{Analytic geometry problem 4 solution continued}
	\begin{block}{Solution}
		We can use the formula for distance from a point to a line:
		\begin{align*}
			\uncover<+->{D &= \frac{\abs{Ax_0 + By_0 + C}}{\sqrt{A^2 + B^2}}\\}
			\uncover<+->{2 &= \frac{\abs{2(0) + 1(0) - 2k}}{\sqrt{2^2 + 1^2}}\\}
			\uncover<+->{2 &= \frac{\abs{-2k}}{\sqrt{5}}\\}
			\uncover<+->{k &= \pm\sqrt{5}}
		\end{align*}
	\end{block}
\end{namedframe}
\begin{namedframe}{Analytic geometry problem 4 solution continued}
	\begin{block}{Solution}
		\[k = \pm \sqrt{5}\]
		Next, we find the area:
		\pause
		\[A = \uncover<+->{\frac{1}{2} \times k \times 2k} = \uncover<+->{k^2} = \uncover<+->{(\pm \sqrt{5})^2} = \uncover<+->{5}\]
	\end{block}
\end{namedframe}
