\section{Factoring}
\begin{namedframe}{Divisibility}
	We will now introduce a new notation, which is more of a shortcut.

	If $b$ divides $a$ with no remainder, then we will write $b \divides a$.

	More formally:
	\[b \divides a \equiv a \bmod b = 0\]

	Or:
	\[b \divides a \iff a = bc\]

	Where $a$, $b$, and $c$ and positive integers.

	If $b \divides a$, then $b$ is a factor of $a$.
\end{namedframe}
\begin{namedframe}{Factors}
	\begin{theorem}[The Unique Factorization Theorem]
		Every positive integer has a unique representation as a product of prime numbers.

		That is, for all numbers $n \in \mathbb{Z}^+$:
		
		\[n = p_1^{a_1} \times p_2^{a_2} \times \dots \times p_k^{a_k}\]

		Where $p_i$ is prime, and $a_i$ is a positive integer.
	\end{theorem}
	\pause
	\begin{example}[180]
		\[180 = 2^2 \times 3^2 \times 5\]
		\[180 = 2 \times 2 \times 3 \times 3 \times 5\]
	\end{example}
\end{namedframe}
\begin{namedframe}{Factoring}
	How do we factor a number?
	\sep
	How do we factor a \emph{large} number?
	\sep
	Try this one:
	\begin{block}{RSA-2048}
		\tiny
		\ttfamily
		251959084756578934940271832400483985714292821262040320277771378360436620207075955562640185258807
		844069182906412495150821892985591491761845028084891200728449926873928072877767359714183472702618
		963750149718246911650776133798590957000973304597488084284017974291006424586918171951187461215151
		726546322822168699875491824224336372590851418654620435767984233871847744479207399342365848238242
		811981638150106748104516603773060562016196762561338441436038339044149526344321901146575444541784
		240209246165157233507787077498171257724679629263863563732899121548314381678998850404453640235273
		81951378636564391212010397122822120720357
	\end{block}
	\pause
	This number has two factors. Nobody knows what they are.

	There was a $\$\num{200000}$ prize to factor this number.
	People had over 15 years to factor it, but nobody was able to before the contest period ended.
\end{namedframe}
\begin{namedframe}{What is the greatest common divisor?}
	The greatest common divisor (GCD) of two numbers is the largest number that divides both numbers.
	\sep
	More formally:
	\[x = \gcd(a, b) \such a, b \in \mathbb{Z}\]
	Where $x$ is the largest number such that:
	\[x \divides a \wedge x \divides b\]
	\pause
	If two numbers are coprime, their $\gcd$ is $1$.
\end{namedframe}
\begin{namedframe}{How do we find the greatest common divisor?}
	How do we find the greatest common divisor?
	\sep
	We could list all the factors, and the biggest one would be the $\gcd$.

	But, as we saw above, factoring is very hard.
	Is there a better way?
	\sep
	Of course. Otherwise I wouldn't ask.
\end{namedframe}
\begin{namedframe}{Euclidean algorithm}
	Thousands of years ago, Euclid came up with an algorithm to find the $\gcd$.
	\begin{block}{Euclidean algorithm}
		To find $\gcd(a, b)$, do the following:
		\begin{enumerate}
			\item Let $r_0 = a$, $r_1 = b$, and $i = 1$.
			\item If $r_i = 0$ then $\gcd(a, b) = r_i$.
			\item Write $r_{i-1} = q_ir_i + r_{i+1}$ and increment $i$ by $1$.
			
			      Here, $r_{i+1} = r_i \bmod r_{i-1}$.
			\item Go back to step 2.
		\end{enumerate}
	\end{block}
\end{namedframe}
\begin{namedframe}{Extended Euclidean algorithm}
	The extended Euclidean algorithm is essentially the Euclidean algorithm in reverse.

	We use substitution while working backwards.

	It allows us to find two integers, $x$ and $y$, that satisfy:
	\[\gcd(a, b) = ax + by\]
	\pause
	When $\gcd(a, b) = 1$, $x = a^{-1} \mod b$, where $a^{-1}$ is the inverse of $a$ in modulus $b$.
\end{namedframe}
\begin{namedframe}{Extended Euclidean algorithm example}
	Find the inverse of $3$ in modulus $26$.
	\begin{align*}
		26 &= (8)3 + 2\\
		3  &= (1)2 + 1\\
		2  &= (2)1 + 0
	\end{align*}
	\pause
	\[1 = 3 - (1)2 = 3 - (26 - (8)3)\]
	\[1 = (9)3 - 26\]
	\pause
	And so the inverse of $3$ in $\bmod\ 26$ is $9$.
	
	We can verify this:
	\[(3 \times 9) \bmod 26 = 1\]
\end{namedframe}
