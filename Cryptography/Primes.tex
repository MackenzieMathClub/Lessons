\section{Primes}
\begin{namedframe}{What is a prime number?}
	A \emph{prime number} is a positive integer that is only divisible by $1$ and itself.
	\begin{examples}
		\begin{compactmath}
			\mathbb{P} = \{2, 3, 5, 7, 11, 13, 17, \dots\}
		\end{compactmath}
	\end{examples}
	If an integer greater than $1$ is not prime, it is called a \emph{composite number}.
	\vertspace
	$1$ is special, and is called the \emph{unit number}
	\sep
	\begin{proof}[Proof $1$ is not prime.]
		In the past, some mathematicians said that $1$ is prime.
		All of them are dead now.
		\[\therefore 1 \notin \mathbb{P} \qedhere\]
	\end{proof}
\end{namedframe}
\begin{namedframe}{The largest prime number}
	The largest known prime number\footnote{As of January 5th, 2018} is:
	\[M_{\num{77232917}} = 2^{\num{77232917}} - 1\]
	If you were to print this number out, it would be 6055 pages long!
	\vertspace
	This prime was discovered by Jonathan Pace on December 26, 2017 after 6 days of continuous computer computations.
	The discovery was published on January 3, 2018.
	\sep
	This number is a \emph{Mersenne prime}.
	These are prime numbers of the form $2^n - 1$, and we label these primes as $M_n$ for short.
	\sep
	What's special and useful about Mersenne primes?
	\pause
	Not much.
\end{namedframe}
\begin{namedframe}{How many primes are there?}
	Is the number of primes finite?
	\sep
	No! There are infinite prime numbers!

	This was proved thousands of years ago by Euclid.
\end{namedframe}
\begin{namedframe}{Proof of infinite primes}
	Assume the list of primes is finite, and there are only $n$ prime numbers. We will call our list of prime numbers $P$.
	\[P = \{p_1, p_2, \dots , p_{n-1}, p_n\}\]
	Where $p_k$ is the $k$th prime number.
	\sep
	Now, let $m$ be the product of all numbers in $P$ plus $1$.
	\[m = (p_1 \times p_2 \times \dots \times p_{n-1} \times p_n) + 1 = \left(\sum_{i=1}^n p_i\right) + 1\]
	$m$ is either prime or not prime.
	Let's look at both cases.
\end{namedframe}
\begin{namedframe}{Proof of infinite primes: $m$ is prime}
	First, let's consider the case that $m$ is prime.
	\sep
	Note that $m$ is not in our original list, $P$.
	\sep
	If $m$ is prime, our original list is incomplete, and there are more prime numbers than we listed.
\end{namedframe}
\begin{namedframe}{Proof of infinite primes: $m$ is not prime}
	If $m$ is not prime, then it must be divisible by a prime number.

	Notice that $m$ cannot be divisible by any numbers in $P$, as they would not divide a number that is a multiple of themselves plus $1$.
	\sep
	For example:
	\vspace{-1ex}
	\[P = \{2,3,5,7,11,13\}\]
	\vspace{-3ex}
	\[m = 2 \times 3 \times 5 \times 7 \times 11 \times 13 + 1 = \num{30031}\]
	\varsep{-5ex}
	\begin{center}
		\begin{minipage}{0.45\textwidth}
			\begin{align*}
				\num{30031} \bmod 2 &= 1\\
				\num{30031} \bmod 3 &= 1\\
				\num{30031} \bmod 5 &= 1
			\end{align*}
		\end{minipage}
		\begin{minipage}{0.45\textwidth}
			\begin{align*}
				\num{30031} \bmod 7  &= 1\\
				\num{30031} \bmod 11 &= 1\\
				\num{30031} \bmod 13 &= 1
			\end{align*}
		\end{minipage}
	\end{center}
	\varvsep{-2ex}
	Here, we can see that since $\num{30031}$ is a multiple plus $1$ of every number in $P$, no numbers in $P$ will divide it.
	\pause
	But if $\num{30031}$ is not prime, then it divisible by a prime number, so there must be some prime numbers missing from our original list.
	\pause
	$\num{30031}$ is divisible by $59$ and $509$, so these numbers are missing from our list.
\end{namedframe}
\begin{namedframe}{Primality}
	How do we check if a number is prime?
	\sep
	How do we check if a number is prime \emph{quickly}?
	\sep
	With a \emph{very} fast computer. Algorithms exist (some of which run in polynomial time) but they are \emph{very} slow.

	Here, ``quickly'' means the computer will finish before we die.
\end{namedframe}
\begin{namedframe}{This fits in the margins}
	\begin{theorem}[Fermat's Little Theorem]
		Let $p$ be a prime number, and $a$ an integer that does not divide $p$.
		\vertspace
		Then:
		\[a^{p-1} \bmod p = 1\]
	\end{theorem}
	\pause
	\begin{theorem}[Euler-Fermat Generalization]
		Fermat's Little Theorem can be generalized as:
		\[a^{\phi(n)} \bmod n = 1\]
		Where $\phi(n)$ is Euler's totient function, which gives us the number of integers less than or equal to $n$ that are coprime to $n$.
	\end{theorem}
	\pause
	For an extra challenge, prove the \emph{Euler-Fermat Generalization} using \emph{Fermat's Little Theorem}.
\end{namedframe}
