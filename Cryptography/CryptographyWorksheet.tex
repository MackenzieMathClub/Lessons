\documentclass[letterpaper,12pt]{article}

\usepackage[margin=1in]{geometry}
\usepackage{microtype}
\usepackage{parskip}
\usepackage{tasks}
\usepackage{fancyhdr}
\usepackage{titling}

% Typography and font packages.
\usepackage{lmodern}
\usepackage{microtype}

% Math packages.
\usepackage{amsmath}
\usepackage{amssymb}
\usepackage{mathtools}
\usepackage{commath}
\usepackage{siunitx}

\frenchspacing

% Macros for math
\newcommand{\such}{\mid}
\newcommand{\real}{\mathbb{R}}
\newcommand{\integer}{\mathbb{Z}}
\newcommand{\nat}{\mathbb{N}}
\newcommand{\complex}{\mathbb{C}}
\DeclarePairedDelimiter{\ceil}{\lceil}{\rceil}
\DeclarePairedDelimiter{\floor}{\lfloor}{\rfloor}
\newcommand{\restrict}[1]{\ \left\{#1\right\}}

% Unit setup
\NewDocumentCommand{\varSI}{}{\SI[number-math-rm=\mathnormal,parse-numbers=false]} % Use variables as the value of units.


\newcommand{\solutionspace}[1][1in]{\vspace{#1}~\newline}

% Redefine the page style.
\pagestyle{fancy}
\renewcommand{\headrulewidth}{1pt}
\renewcommand{\footrulewidth}{0.4pt}
\lhead{\theauthor}
\chead{\LARGE\thetitle}
\rhead{\thedate}
\cfoot{Page \thepage}
\lfoot{\texttt{mackenziemathclub.github.io}}


\usepackage{amsthm}

\title{Cryptography}
\author{Mackenzie Math Club}
\date{January 8, 2018}

\rfoot{\parbox[t]{0.35\textwidth}{\copyright{} Caroline Liu, Vincent Macri, and Samantha Unger, 2018}}

\newcommand{\divides}{\operatorname{|}}

\theoremstyle{definition}
\newtheorem*{fermatLittle}{Fermat's Little Theorem}
\newtheorem*{eulerFermat}{Euler-Fermat Generalization}
\newtheorem*{uniqueFactorization}{Unique Factorization Theorem}

\begin{document}
	\section*{Modular Arithmetic}
		\emph{Mod} is the remainder of two numbers.
		\[a \bmod b = \text{the remainder of } a \div b\]
		Mod can be easily computed with the following formula:
		\[a \bmod b = a - b \floor*{\frac{a}{b}}\]
		\subsection{Inverse modulo}
			For modulus $n$, $b$ is the inverse of $a$ when:
			\[a \times b \mod n = 1 \such 0 < a, b < n\]
			The inverse of $a$ exists if and only if $a$ and $n$ are coprime. That is, $\gcd(a, n) = 1$.
	\section*{Primes}
		\begin{fermatLittle}
			Let $p$ be a prime number, and $a$ an integer that does not divide $p$.

			Then:
			\[a^{p-1} \bmod p = 1\]
		\end{fermatLittle}
		\begin{eulerFermat}
			Fermat's Little Theorem can be generalized as:
			\[a^{\phi(n)} \bmod n = 1\]
			Where $\phi(n)$ is Euler's totient function, which gives us the number of integers less than or equal to $n$ that are coprime to $n$.
		\end{eulerFermat}
		\textbf{Problem 1}: Try and prove the Euler-Fermat generalization using Fermat's Little Theorem:
		\solutionspace{1in}
	\section*{Factoring}
		If $b$ divides $a$ with no remainder, we will write:
		\[b \divides a\]
		Read as ``$b$ divides $a$''
		\begin{uniqueFactorization}
			Every positive integer has a unique representation as a product of prime numbers.

			That is, for all numbers $n \in \mathbb{Z}^+$:
			
			\[n = p_1^{a_1} \times p_2^{a_2} \times \dots \times p_k^{a_k}\]

			Where $p_i$ is prime, and $a_i$ is a positive integer.
		\end{uniqueFactorization}
		\subsection*{Greatest common divisor}
			\[x = \gcd(a, b) \such a, b \in \mathbb{Z}\]
			Where $x$ is the largest number such that:
			\[x \divides a \wedge x \divides b\]
			Two numbers are coprime (only share $1$ as a factor) when their $\gcd$ is $1$.
		\subsection*{Euclidean algorithm}
			To find $\gcd(a, b)$, do the following:
			\begin{enumerate}
				\item Let $r_0 = a$, $r_1 = b$, and $i = 1$.
				\item If $r_i = 0$ then $\gcd(a, b) = r_i$.
				\item Write $r_{i-1} = q_ir_i + r_{i+1}$ and increment $i$ by $1$.
				
				      Here, $r_{i+1} = r_i \bmod r_{i-1}$.
				\item Go back to step 2.
			\end{enumerate}
			The extended form of the algorithm lets us find two integers, $x$ and $y$, such that:
			\[\gcd(a,b) = ax + by\]
			We can use this to find the inverse of modular multiplication.
	\section*{RSA}
		\subsection*{Key generation}
			The first step of RSA encryption is to generate a public-private keypair.
			\begin{enumerate}
				\item Start by picking two random prime numbers, $p$ and $q$, that are similar in size. The bigger the better.
				\item Calculate:
					\begin{enumerate}
						\item $n = p \times q$
						\item $\phi(n) = (p - 1) \times (q - 1)$
					\end{enumerate}
				\item Next, choose a random integer $e$ such that $0 < e < \phi(n)$ and $e$ has an inverse in $\bmod \phi(n)$ ($e$ and $\phi(n)$ are coprime).
				\item For the final step, compute $d$ to be the inverse of $e$. $e \times d \mod \phi(n) = 1 \such 0 < d < \phi(n)$
			\end{enumerate}
			Your public keypair is $(n, e)$. Your private key is $d$.
		\subsection*{Encrypting}
			To encrypt a message $m$ such that $m < n$, calculate the \emph{ciphertext}, $c$, as so:
			\[c = m^e \bmod n\]
		\subsection*{Decrypting}
			To decrypt $c$, apply the inverse of raising $m$ to the $e$: raise $m$ to the $d$.
			\[m = r = c^d \bmod n\]
	\section*{Last Problem}
		Encrypt whatever you want using RSA.
\end{document}
