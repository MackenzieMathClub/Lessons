\section{RSA}
\begin{namedframe}{RSA encryption}
	The RSA encryption algorithm was originally classified before being rediscovered by three people with the initials R, S, and A in 1977.
	RSA encryption or a variant of it is used today in many online systems, especially for setting up more permanent encrypted sessions.
	\sep
	Today, we will learn how to encrypt messages using the RSA system.
	\vertspace
	While RSA has not been broken by anyone, there are systems that are considered to be \emph{more} secure because they provide something called ``perfect forward secrecy''. However, a lot of these systems are similar to, or even use RSA.
\end{namedframe}
\begin{namedframe}{RSA overview}
	What is RSA?
	\begin{itemize}[<+(1)->]
		\item Public keys
		\item Private keys
		\item Padlocks
		\item Factoring is hard
	\end{itemize}
	\pause
	Alice and Bob.
\end{namedframe}
\begin{namedframe}{Key generation}
	The first step of RSA encryption is to generate a public-private keypair.
	\begin{enumerate}[<+(1)->]
		\item Alice starts by picking two random prime numbers, $p$ and $q$, that are similar in size. The bigger the better.
		\item Alice then calculates:
			\begin{enumerate}
				\item $n = p \times q$
				\item $\phi(n) = (p - 1) \times (q - 1)$
			\end{enumerate}
		\item Next, Alice chooses a random integer $e$ such that $0 < e < \phi(n)$ and $e$ has an inverse in $\bmod \phi(n)$ ($e$ and $\phi(n)$ are coprime).
		\item For the final step, Alice computes $d$ to be the inverse of $e$. $e \times d \mod \phi(n) = 1 \such 0 < d < \phi(n)$
	\end{enumerate}
	\pause
	Alice's public keypair is $(n, e)$.

	Alice's private key is $d$.
\end{namedframe}
\begin{namedframe}{Encrypting}
	Bob wants to send a message to Alice.
	He has turned his message into a number $m$, such that $m < n$ (if $m \geq n$, then Bob will split the message up into multiple short messages).

	If $m$ and $n$ are not coprime, then one could easily factorize $n$, thus breaking the encryption.
	\sep
	Bob then downloads Alice's public key from somewhere he trusts and calculates the \emph{ciphertext}, $c$:
	\[c = m^e \bmod n\]
	And he sends the encrypted message, $c$ to Alice.
\end{namedframe}
\begin{namedframe}{Decrypting}
	To decrypt the message, Alice calculates calculates $r$, which is equal to $m$, Bob's message:
	\[r = c^d \bmod n\]
	\sep
	How do we know that $r = m$?
\end{namedframe}
\begin{namedframe}{Euler's theorem}
	\begin{theorem}[Euler's theorem]
		When $a$ and $n$ are coprime positive integers:
		\[a^{\phi(n)} \bmod n = 1\]
	\end{theorem}
\end{namedframe}
\begin{namedframe}{Proof that $r = m$}
	Remember that $\left(m^e\right)^d = m^{ed}$. So:
	\[r = m^{ed} \bmod n\]
	We know that $ed = 1 \bmod \phi(n)$ because we chose $e$ and $d$ to have that property.

	This means that an integer, $q$, exists such that:
	\[ed = q\phi(n) + 1\]
\end{namedframe}
