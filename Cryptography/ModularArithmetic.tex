\section{Modular Arithmetic}
\begin{namedframe}{Quick review}
	We define the \alert{$\bmod$} operator as being the remainder when dividing two numbers.
	That is:
	\[a \bmod b = \text{the remainder of } a \div b\]
	In some programming languages, modulo is written as \alert{$\%$} or \alert{$\operatorname{rem}$}.
	Use whichever notation you are most comfortable with.
	\begin{examples}
		\centering
		\begin{minipage}{0.45\textwidth}
			\[4 \bmod 2 = 0\]
			\[5 \bmod 2 = 1\]
		\end{minipage}
		\begin{minipage}{0.45\textwidth}
			\[7 \bmod 3 = 1\]
			\[9 \bmod 5 = 4\]
		\end{minipage}
	\end{examples}
	\footnotesize
	The definition of modulo (mod for short) is a bit trickier with negative numbers.
	It also doesn't matter for today, as we're only looking at mod with positive numbers.
\end{namedframe}
\begin{namedframe}{Calculating modulo}
	While we could do long division to find the remainder when we want to calculate modulo, I prefer to use this formula:
	\[a \bmod b = a - b \floor*{\frac{a}{b}}\]
	Where $\floor{x}$ is the floor function, which rounds a number \emph{down} to an integer.
	\sep
	There are other method, but I think this one is the hardest to mess up.
	Use whatever method you are most comfortable with.
\end{namedframe}
\begin{namedframe}{Inverse of modular multiplication}
	How do we do division in modular arithmetic?
	\sep
	Division is the inverse of multiplication.
	\sep
	Recall from the \emph{group theory} lesson that the identity element in multiplication is $1$.
	\sep
	So, for modulus $n$, $b$ is the inverse of $a$ when:
	\[a \times b \mod n = 1 \such 0 < a, b < n\]
	\pause
	Not all numbers have an inverse in modular arithmetic.
	\vertspace
	It turns out $a$ has an inverse in $\bmod\ n$ if and only if $a$ and $n$ are coprime.
	\sep
\end{namedframe}
\begin{namedframe}{Inverse of modular multiplication examples}
	\[3 \times 7 \mod 20 = 1\]
	Here, $7$ is the inverse of $3$, and $3$ is the inverse of $7$.
	\sep
	$2$ does \emph{not} have an inverse modulo $4$.
	\[2 \times b \mod 4 = 1\]
	There is no integer value for $b$ that satisfies this equation.
	\sep
	We will soon learn the algorithm to find the inverse to modular multiplication.
\end{namedframe}
