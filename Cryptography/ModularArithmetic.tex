\section{Modular Arithmetic}
\begin{namedframe}{Quick review}
	We define the \alert{$\bmod$} operator as being the remainder when dividing two numbers.
	That is:
	\[a \bmod b = \text{the remainder of } a \div b\]
	In some programming languages, modulo is written as \alert{$\%$} or \alert{$\operatorname{rem}$}.
	Use whichever notation you are most comfortable with.
	\begin{examples}
		\centering
		\begin{minipage}{0.45\textwidth}
			\[4 \bmod 2 = 0\]
			\[5 \bmod 2 = 1\]
		\end{minipage}
		\begin{minipage}{0.45\textwidth}
			\[7 \bmod 3 = 1\]
			\[9 \bmod 5 = 4\]
		\end{minipage}
	\end{examples}
	\footnotesize
	The definition of modulo (mod for short) is a bit trickier with negative numbers.
	It also doesn't matter for today, as we're only looking at mod with positive numbers.
\end{namedframe}
\begin{namedframe}{Divisibility}
	We will also introduce a new notation, which is more of a shortcut.

	If $b$ divides $a$ with no remainder, then we will write $b \divides a$.

	More formally:
	\[b \divides a \equiv a \bmod b = 0\]

	Or:
	\[b \divides a \iff a = bc\]

	Where $a$, $b$, and $c$ and positive integers.
\end{namedframe}
