\section{Introduction to Cryptography}
\begin{namedframe}{What are factors used for?}
	Factorization of numbers is very useful in cryptography.

	The reason for this is that factoring large numbers takes a \emph{very} long time, but the maths for checking factorization are quick.

	We can use this to develop a way to encode messages so they can only be read by certain people.
	This is called cryptography.
\end{namedframe}
\begin{namedframe}{What is cryptography}
	Simply put, cryptography is the study of ways to encrypt messages.
	\vertspace
	Encryption is when you transform a message so that it cannot easily be read by someone without a key.
	Encryption is like a lock, but instead of locking your house, it locks information.
	\vertspace
	The use of encryption goes back thousands of years.
\end{namedframe}
\begin{namedframe}{The Caesar cipher}
	One example of encryption was used by Julius Caesar to keep military messages protected from spies.

	Caesar's encryption worked like this:
	\sep
	Pick a number, $n$, between $1$ and $25$.

	Shift every letter in the message that many letters to the right, wrapping around when you reach z.
	\sep
	For example, with $n = 3$:

	\texttt{Plaintext message: `Crypto is fun!'}

	\texttt{Encrypted message: `Fubswr lv ixq!'}

	Caesar's generals knew what value for $n$ Caesar used, and would reverse the process to decode his messages.
	\sep
	Obviously, this isn't very secure. Why?
\end{namedframe}
