\section{Group Definition}
\begin{namedframe}{Definition}
	A group is made of a set and an operation that acts on the elements of that set. In general, we denote a group as $(G, \cdot)$, where $G$ is the set, and $\cdot$ is the operation.

	A group also satisfies the four group axioms: identity, invertibility, associativity, and closure.
\end{namedframe}
\begin{namedframe}{The identity axiom}
	The identity axiom says that:

	There exists an element $e \in G$, called the identity, such that:
	\[a \cdot e = e \cdot a = a\]
	for all $a \in G$.
\end{namedframe}
\begin{namedframe}{The invertibility axiom}
	The invertibility axiom states that for all $a \in G$, there exists $a^{-1}$ such that:
	\[a \cdot a^{-1} = e\]
	We refer to $a^{-1}$ as the inverse of $a$.
\end{namedframe}
\begin{namedframe}{The associativity axiom}
	The associativity axiom states that for all $a, b, c \in G$:
	\[(a \cdot b) \cdot c = a \cdot (b \cdot c)\]
\end{namedframe}
\begin{namedframe}{The closure axiom}
	The closure axiom states that for all $a, b \in G$ where $a \cdot b = c$, $c \in G$.
	\vertspace
	Or, performing the group's operation on any two elements in $G$ will yield a third element also in $G$.
\end{namedframe}
\begin{namedframe}{The power of sets}
	Now, behold the almighty power of sets as we are about to simultaneously prove infinite things.
\end{namedframe}
