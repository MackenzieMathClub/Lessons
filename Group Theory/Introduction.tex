\section{Pure Math}
\begin{namedframe}{What is pure math?}
	\begin{itemize}
		\item Math for the sake of math
		\item Math is art
		\item Math is beautiful
		\item Accidental applications
		\item Chemistry is gross
	\end{itemize}
\end{namedframe}
\section{The History of Group Theory}
\begin{namedframe}{A quote}
	\textquote[Sir Arthur Stanley Eddington]{We need a super-mathematics in which the operations are as unknown as the quantities they operate on, and a super-mathematician who does not know what he is doing when he performs these operations. Such a super-mathematics is the Theory of Groups.}
	\sep
	In other words, group theory is something so powerful that a proof using group theory will usually prove something in many other branches of mathematics at the same time.
\end{namedframe}
\begin{namedframe}{The quadratic formula}
	For $ax^2 + bx + c = 0$:
	\[x = \frac{-b \pm \sqrt{b^2 - 4ac}}{2a}\]
	There also exist similar formulas for cubic and quartic functions, but they are disgusting and won't fit on this slide. But they do exist, which means a computer can easily find the roots of cubic and quartic functions.
	\sep
	Is there a formula for quintics, sextics, or for any $n$ degree polynomial?
	\sep
	No! And this was proved by \'{E}variste Galois using group theory.
\end{namedframe}
\begin{namedframe}{A quote on group theory in science}
	\textquote[Irving Adler]{The importance of group theory was emphasized very recently when some physicists using group theory predicted the existence of a particle that had never been observed before, and described the properties it should have. Later experiments proved that this particle really exists and has those properties.}
	\sep
	Keeping this quote in mind, let's look at what a group is.
\end{namedframe}
