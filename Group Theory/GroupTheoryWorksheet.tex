\documentclass[letterpaper,12pt]{article}

\usepackage[margin=1in]{geometry}
\usepackage{microtype}
\usepackage{parskip}
\usepackage{tasks}
\usepackage{fancyhdr}
\usepackage{titling}

% Typography and font packages.
\usepackage{lmodern}
\usepackage{microtype}

% Math packages.
\usepackage{amsmath}
\usepackage{amssymb}
\usepackage{mathtools}
\usepackage{commath}
\usepackage{siunitx}

\frenchspacing

% Macros for math
\newcommand{\such}{\mid}
\newcommand{\real}{\mathbb{R}}
\newcommand{\integer}{\mathbb{Z}}
\newcommand{\nat}{\mathbb{N}}
\newcommand{\complex}{\mathbb{C}}
\DeclarePairedDelimiter{\ceil}{\lceil}{\rceil}
\DeclarePairedDelimiter{\floor}{\lfloor}{\rfloor}
\newcommand{\restrict}[1]{\ \left\{#1\right\}}

% Unit setup
\NewDocumentCommand{\varSI}{}{\SI[number-math-rm=\mathnormal,parse-numbers=false]} % Use variables as the value of units.


\newcommand{\solutionspace}[1][1in]{\vspace{#1}~\newline}

% Redefine the page style.
\pagestyle{fancy}
\renewcommand{\headrulewidth}{1pt}
\renewcommand{\footrulewidth}{0.4pt}
\lhead{\theauthor}
\chead{\LARGE\thetitle}
\rhead{\thedate}
\cfoot{Page \thepage}
\lfoot{\texttt{mackenziemathclub.github.io}}


\title{Group Theory}
\author{Mackenzie Math Club}
\date{December 11, 2017}

\rfoot{\parbox[t]{0.35\textwidth}{\copyright{} Vincent Macri, 2017}}

\begin{document}
	\section{Quaternions}
		For this section, consider the group $(\mathcal{Q}_8, \times)$, where $\mathcal{Q}_8$ is the set of quaternion elements. That is, $\mathcal{Q}_8 \coloneqq \{-1, 1, -i, i, -j, j, -k, k\}$, and multiplication has the following additional rules:
		\[i^2 = j^2 = k^2 = -1 \qquad ij = k\]
		\vspace{-10ex}
		\subsection{$ji$}
			What is $ji$? Remember, multiplication is not commutative for quaternions, so $ji \neq k$!
			\solutionspace{1ex}
		\subsection{Multiplication table}
			Draw out the multiplication table for this group:
			\begin{equation*}
				\begin{array}{r|c|c|c|c|c|c|c|c}
					\times & \mathmakebox[1cm]{-1} & \mathmakebox[1cm]{1} & \mathmakebox[1cm]{-i} & \mathmakebox[1cm]{i} & \mathmakebox[1cm]{-j} & \mathmakebox[1cm]{j} & \mathmakebox[1cm]{-k} & \mathmakebox[1cm]{k}\\\hline
					-1     &    &   &    &   &    &   &    &  \\\hline
					1      &    &   &    &   &    &   &    &  \\\hline
					-i     &    &   &    &   &    &   &    &  \\\hline
					i      &    &   &    &   &    &   &    &  \\\hline
					-j     &    &   &    &   &    &   &    &  \\\hline
					j      &    &   &    &   &    &   &    &  \\\hline
					-k     &    &   &    &   &    &   &    &  \\\hline
					k      &    &   &    &   &    &   &    &
				\end{array}
			\end{equation*}
	\section{Subgroups of $(\mathbb{Z}, +)$}
		\subsection{$2\mathbb{Z}$}
			We will say that the subset $2\mathbb{Z} \subset \mathbb{Z}$, where $2\mathbb{Z} \coloneqq \{\dots, -6, -4, -2, 0, 2, 4, 6, \dots\}$. That is, $2\mathbb{Z}$ is the set of even integers.

			We will call $(2\mathbb{Z}, +)$ a subgroup of $(\mathbb{Z}, +)$ is $(2\mathbb{Z}, +)$ is also a group. Is it a group? Show that it either does or doesn't satisfy all four group axioms.
		\subsection{$\{-4, -2, 0, 2, 4\}$}
			Is $(\{-4, -2, 0, 2, 4\}, +)$ a subgroup of $(\mathbb{Z}, +)$? Show that it either does or doesn't satisfy all four group axioms.
	\newpage
	\section{Order of a group}
		Similar to the cardinality of a set $S$, the order of a group $(G, \cdot)$ is defined as the number of elements in $G$. We write this as $|G|$.

		\subsection{$(\mathbb{Z}_{12}, +)$}
			For the group $(\mathbb{Z}_{12}, +)$, what is $|\mathbb{Z}_{12}|$?
	\section{Equivalent groups}
		Fill out the follow multiplication tables.

		While they are called multiplication tables, we still use the group's operation, which may or may not be multiplication.
		\subsection{$(\mathbb{Z}_6^*, \times)$}
			Remember, $\mathbb{Z}_6^* \coloneqq \{1, 5\}$.
			\begin{equation*}
				\begin{array}{r|c|c}
					\times & \mathmakebox[1cm]{1} & \mathmakebox[1cm]{5}\\\hline
					1      &                      &                     \\\hline
					5      &                      &
				\end{array}
			\end{equation*}
		\subsection{$(\mathbb{Z}_2, +)$}
			Remember, $\mathbb{Z}_2 \coloneqq \{0, 1\}$.
			\begin{equation*}
				\begin{array}{r|c|c}
					+ & \mathmakebox[1cm]{0} & \mathmakebox[1cm]{1}\\\hline
					0 &                      &                     \\\hline
					1 &                      &
				\end{array}
			\end{equation*}
		\subsection{$(\{a,b\}, \cdot)$}
			You may have noticed a pattern by now for groups with an order of $2$. Without knowing the operation or the elements, fill in the following:
			\begin{equation*}
				\begin{array}{r|c|c}
					\cdot & \mathmakebox[1cm]{a} & \mathmakebox[1cm]{b}\\\hline
					a     &                      &                     \\\hline
					b     &                      &
				\end{array}
			\end{equation*}
			Remember that a group must have an identity. Based on how you filled the above tables, is $a$ or $b$ the identity element in this final question?
\end{document}
