\part{Functions and Equations}\label{part:functionsAndEquations}
\frame{\partpage}
\section{Review}
\subsection{Parabolas}
\begin{namedframe}{Quadratic formula}
	For the quadratic function $f(x) = ax^2 + bx + c$ with $a, b, c \in \real \such a \neq 0$, there are two zeroes (roots) given by the quadratic formula:
	\[x = \frac{-b \pm \sqrt{b^2 - 4ac}}{2a}\]
	Where the \emph{discriminant} ($\Delta$) is the value inside the square root:
	\[\Delta = b^2 - 4ac\]
	These roots are:
	\begin{description}
		\item[Real and distinct] if the 
		\item[Real and equal]
		\item[Non-real and distinct]
	\end{description}
\end{namedframe}
