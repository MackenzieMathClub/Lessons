\documentclass[letterpaper,12pt]{article}

\usepackage[margin=1in]{geometry}
\usepackage{amsthm}
\usepackage{parskip}
\usepackage{tasks}
\usepackage{fancyhdr}
\usepackage{titling}

% Typography and font packages.
\usepackage{lmodern}
\usepackage{microtype}

% Math packages.
\usepackage{amsmath}
\usepackage{amssymb}
\usepackage{mathtools}
\usepackage{commath}
\usepackage{siunitx}

\frenchspacing

% Macros for math
\newcommand{\such}{\mid}
\newcommand{\real}{\mathbb{R}}
\newcommand{\integer}{\mathbb{Z}}
\DeclarePairedDelimiter{\ceil}{\lceil}{\rceil}
\DeclarePairedDelimiter{\floor}{\lfloor}{\rfloor}

% Unit setup
\NewDocumentCommand{\varSI}{}{\SI[number-math-rm=\mathnormal,parse-numbers=false]} % Use variables as the value of units.


\newcommand{\solutionspace}[1]{\vspace{#1}~\newline}

% Redefine the page style.
\pagestyle{fancy}
\renewcommand{\headrulewidth}{1pt}
\renewcommand{\footrulewidth}{0.4pt}
\lhead{\theauthor}
\chead{\LARGE\thetitle}
\rhead{\thedate}
\cfoot{Page \thepage}
\lfoot{\texttt{mackenziemathclub.github.io}}

\newcommand{\TRUE}{\mathrm{TRUE}}
\newcommand{\FALSE}{\mathrm{FALSE}}
\newcommand{\NOT}{\mathrm{NOT}}
\newcommand{\AND}{\mathrm{AND}}
\newcommand{\IFTHENELSE}{\mathrm{IFTHENELSE}}
\newcommand{\XOR}{\mathrm{XOR}}
\newcommand{\SUCC}{\mathrm{SUCC}}


\title{Lambda Calculus}
\author{Mackenzie Math Club}
\date{April 30, 2018}

\rfoot{\begin{minipage}[t]{0.35\textwidth}\raggedleft\copyright{} Caroline Liu, Vincent Macri, and Samantha Unger, 2018\end{minipage}}

\begin{document}
	\section*{Definitions}
		\begin{minipage}{0.5\textwidth}
			\subsection*{Booleans}
			$\TRUE = \lambda xy.x$\\
			$\FALSE = \lambda xy.y$\\
			$\NOT = \lambda b . b (\FALSE\ \TRUE)$\\
			$\AND = (\lambda pq . p) (q\ p)$\\
			$\IFTHENELSE = (\lambda btf . b) (t\ f)$
		\end{minipage}
		\begin{minipage}{0.5\textwidth}
			\subsection*{Church Numerals}
			$0 = \lambda fx.x$\\
			$1 = \lambda fx.f(x)$\\
			$2 = \lambda fx.f(f(x))$\\
			$n = \lambda fx.f^n (x)$\\
			$\SUCC = \lambda nfx.f (nfx)$
		\end{minipage}
	\section{One Argument}
		\subsection{Write the function$f(a, b) = a^2 + b^2$ as a lambda calculus expression}
		\solutionspace{1in}
	\section{Booleans}
		\subsection{Write a lambda calculus expression for $\NAND$}
		\solutionspace{1in}
		\subsection{Write a lambda calculus expression for $\XOR$}
		\solutionspace{1in}
	\section{Church Numerals}
		\subsection{What is the numerical value of the Church numeral whose lambda expression is $\lambda fx.f (f(f(f(f(x)))))$?}
		\solutionspace{1ex}
		\subsection{What is the lambda expression of the Church numeral whose numerical value is $7$?}
		\solutionspace{1ex}
		\subsection{Compute $\SUCC\ \lambda fx.f(f(f(x)))$ and write its numerical value.}
\end{document}
