\section{One Parameter}
\begin{namedframe}{Currying}
	How would we define a function that outputs $x + y$ in math class?
	\pause
	\[f(x, y) = x + y\]
	\pause
	In lambda calculus, functions are only allowed to have \alert{one} parameter.
	\sep
	So, to add two numbers, we have a function output another function, like this:
	\[\lambda x . \lambda y . x + y\]
	And we use it like this:
	\[(\lambda x . \lambda y . x + y) (2, 3) \pause = (\lambda y . 2 + y) 3 \pause = 2 + 3 = 5\]
\end{namedframe}
\begin{namedframe}{Why one argument?}
	\begin{itemize}[<+->]
		\item Very simple
		\item Very powerful
		\item Functions can only have one variable
	\end{itemize}
\end{namedframe}
\begin{namedframe}{But I'm lazy}
	Me too!
	\sep
	We have some shortcuts to help us write down lambda calculus expressions, but it's important to remember what they represent, without the shortcuts.
	\pause
	\[\lambda x . \lambda y . \lambda z . A\]
	Can be abbreviated as:
	\[\lambda xyz . A\]
	\pause
	Also, we assume that we evaluate a function with ``multiple'' arguments starting with the leftmost parameter.
\end{namedframe}
\begin{namedframe}{This is stupid. It just makes everything harder.}
	\uncover<2->{For those examples, }you're right!

	\uncover<3->{Let's get to the fun stuff now!}
\end{namedframe}
