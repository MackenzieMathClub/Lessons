\section{Introduction}
\begin{namedframe}{What is lambda calculus?}
	\begin{itemize}[<+->]
		\item Created by Alonzo Church
		\item A way of representing \alert{pure} mathematical functions
		\item Can represent any computer program
		\item Equivalent to Turing machines
	\end{itemize}
\end{namedframe}
\begin{namedframe}{$x+1$}
	In math class, we would define a function that accepts an argument $x$ and outputs $x+1$ as so:
	\[f(x) \pause = x + 1\]
	\pause
	In lambda calculus, we do it like this:
	\[\lambda \pause x \pause . \pause x + 1\]
	\pause
	If we wanted to find $4 + 1$, we could do this:
	\[f(4) = 4 + 1 = 5\]
	\pause
	In lambda calculus, we \alert{apply} a value to a function like this:
	\[(\lambda x . x + 1) 4 = 4 + 1 = 5\]
	\pause
	You can think of $\lambda$ as $f$, and $.$ as $=$.
\end{namedframe}
