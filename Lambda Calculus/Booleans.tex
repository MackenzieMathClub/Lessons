\section{Booleans}
\begin{namedframe}{Boolean logic}
	\begin{block}{Quote}
		``Any program can be written in lambda calculus.''
		\sep
		\hspace{18em} --- Me, 5 minutes ago
	\end{block}
	\pause
	So, let's bring on the Booleans!
\end{namedframe}
\begin{namedframe}{True and false}
	We define $\true$ as:
	\[\true = \lambda x.\lambda y.x = \lambda xy.x\]
	\pause
	And $\false$ as:
	\[\false = \lambda x.\lambda y.y = \lambda xy.y\]
	\pause
	So $\true$ returns the first value, and $\false$ returns the second.
	\sep
	We will use $\true$ and $\false$ and shorthand for their respective definitions shown here.
\end{namedframe}
\begin{namedframe}{$\not$}
	$\not$ can be written in lambda calculus as:
	\[\not = \lambda b . b (\false\ \true)\]
	\pause
	\begin{exampleblock}{$\not\ \true$}
		\begin{align*}
			\uncover<+->{\left( \lambda b . b (\false\ \true) \right) \true &= \uncover<+->{\true (\false\ \true)}}\\
			\uncover<+->{                                      &= \lambda xy.x (\false\ \true)}\\
			\uncover<+->{                                      &= \false}
		\end{align*}
	\end{exampleblock}
\end{namedframe}
