\section{Booleans}
\begin{namedframe}{Boolean logic}
	\begin{block}{Quote}
		``Any program can be written in lambda calculus.''
		\sep
		\hspace{18em} --- Me, 5 minutes ago
	\end{block}
	\pause
	So, let's bring on the Booleans!
\end{namedframe}
\begin{namedframe}{$\TRUE$ and $\FALSE$}
	\begin{definition}[$\TRUE$]
		\[\TRUE = \lambda xy.x\]
	\end{definition}
	\pause
	\begin{definition}[$\FALSE$]
		\[\FALSE = \lambda xy.y\]
	\end{definition}
	\pause
	So $\TRUE$ returns the first value, and $\FALSE$ returns the second.
	\sep
	We will use $\TRUE$ and $\FALSE$ as shorthand for these definitions.
\end{namedframe}
\begin{namedframe}{$\NOT$}
	\begin{definition}[$\NOT$]
		\[\NOT = \lambda b . b (\FALSE\ \TRUE)\]
	\end{definition}
	\pause
	\begin{exampleblock}{$\NOT\ \TRUE$}
		\begin{align*}
			\uncover<+->{\left( \lambda b . b (\FALSE\ \TRUE) \right) \TRUE &= \uncover<+->{\TRUE (\FALSE\ \TRUE)}\\}
			\uncover<+->{                                      &= \lambda xy.x (\FALSE\ \TRUE)\\}
			\uncover<+->{                                      &= \FALSE}
		\end{align*}
	\end{exampleblock}
\end{namedframe}
\begin{namedframe}{$\AND$}
	\begin{definition}[$\AND$]
		\[\AND = (\lambda pq . p) (q\ p)\]
	\end{definition}
	\pause
	\begin{exampleblock}{$\AND (\TRUE\ \FALSE)$}
		\begin{align*}
			\uncover<+->{&\hphantom{=} \left( (\lambda pq . p) (q\ p) \right) (\TRUE\ \FALSE)\\}
			\uncover<+->{&= \TRUE (\FALSE\ \TRUE)\\}
			\uncover<+->{&= \FALSE}
		\end{align*}
	\end{exampleblock}
\end{namedframe}
