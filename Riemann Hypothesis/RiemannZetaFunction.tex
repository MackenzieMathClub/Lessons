\section{The Riemann Zeta Function}
\begin{namedframe}{The formula}
	\begin{definition}[The Riemann zeta function]
		\[\zeta(s) = \sum_{n=1}^\infty \frac{1}{n^s} \restrict{s \in \complex \such \only<1>{\Re(s) > 1} \only<2->{\alert{\Re(s) > 1}}}\]
		\[s = \sigma + \imagi t\]
		\pause
		\[\Re(s) = \sigma > 1\]
		\pause
		There is another definition of the function for $\sigma \leq 1$, which is beyond the scope of this lesson.
		\sep
		$\zeta(s)$ is symmetric across the vertical line $\sigma = -\frac{1}{2}$.
	\end{definition}
\end{namedframe}
\begin{namedframe}{Negative one twelfth}
	\[\sum_{n=1}^\infty n \neq -\frac{1}{12}\]
	\pause
	Some people incorrectly believe this because $\zeta(-1) = -\frac{1}{12}$.
	\[\zeta(-1) = \sum_{n=1}^\infty \frac{1}{n^{-1}} = \sum_{n=1}^\infty n\]
	\pause
	But, remember that $\Re(s) > 1$, so here we cannot use the simple definition of the Riemann zeta function.
	\sep
	\alert{The sum of the natural numbers is related to, but not equal to, $-\frac{1}{12}$.}
\end{namedframe}
