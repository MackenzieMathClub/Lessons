\subsection{Examples}
\begin{namedframe}{Odd Number Squared}
	\begin{example}
		Given an odd integer $N^2$, prove that $N$ must be an odd integer.	
	\end{example}
	\pause
	Suppose that $N$ is not an odd integer. Thus, it can be expressed as $2M$.
	\pause
	\begin{align*}
		\uncover<+->{N^2 &= (2M)^2\\}
		\uncover<+->{ &= 4M^2\\}
		\uncover<+->{ &= 2(2M^2)}
	\end{align*}
	\uncover<+->{
		$N^2$ is not odd. Therefore, $N$ \textit{is not} not an odd integer, and must be odd.
	}
\end{namedframe}
\begin{namedframe}{Irrationality of $\sqrt{2}$}
	Legend goes that Pythagoras drowned the student that proved the existence of irrational numbers.\pause Quite irrational of him! \pause Moreover, it is believed that the proof of the existence of irrational numbers was done using $\sqrt{2}$. It has also become the classical proof by contradiction.
	\pause
	\begin{example}
		Prove that $\sqrt{2}$ is irrational.	
	\end{example}
	Let's assume that $\sqrt{2}$ is rational.\pause Then, it can be expressed as $\frac{a}{b}$ where $a$ and $b$ are integers and have no common factors. If they have common factors, divide both by the factor until the original condition is met. This has been shown to be possible for any two integers in an earlier lesson.
\end{namedframe}
\begin{namedframe}{Irrationality of $\sqrt{2}$}
	\vspace{-.5cm}
	\begin{align*}
		\uncover<+->{\sqrt{2} &= \frac{a}{b}\\}
		\uncover<+->{\sqrt{2}b &= a\\}
		\uncover<+->{2b^2 &= a^2\\}
	\end{align*}
	
	\uncover<+->{This means that $a$ is even, and can be expressed as $2c$.}
	\begin{align*}
		\uncover<+->{2b^2 &= (2c)^2\\}
		\uncover<+->{b^2 &= 2c^2\\}
	\end{align*}
	\uncover<+->{As before, the above statement implies that $b$ is even.} 	\uncover<+->{However, if both $a$ and $b$ are even, then they share a common factor 2. This contradicts the original assumption. Therefore, $\sqrt{2}$ is irrational.}
\end{namedframe}