\subsection{Examples}
\begin{namedframe}{Odd Number Squared}
	\begin{example}
		Given an integer $N$, prove that $N^2$ must be even, or one greater than even.*	
	\end{example}
	\pause
	Separate the problem into two cases: $N$ is odd, or $N$ is even. Then, you will find that one case will yield an even number (namely $N$ is even) and the other case will yield a number one greater than even (namely $N$ is odd).
\end{namedframe}
\begin{namedframe}{Actual Example}
	\begin{example}
		Given a perfect cube $N^3$, prove that it is always divisible by 9, or 1 greater or 1 less than that.
	\end{example}
	\pause
	For this problem, we will make 3 cases: $N$ is divisible by 3, $N$ is one less than a number divisible by 3, and $N$ is one greater than a number divisible by 3.
\end{namedframe}
\begin{namedframe}{Case 1}
	Case 1 ($N$ is divisible by 3, and can be expressed as $3M$):
	\pause
	\begin{align*}
		N^3 &= (3M)^3\\
			&= 27M^3\\
			&= 9(3M^3)
	\end{align*}
\end{namedframe}
\begin{namedframe}{Case 2}
	Case 2 ($N$ is one less than a number divisible by 3, and can be expressed as $3M-1$):
	\pause
	\begin{align*}
		N^3 &= (3M-1)^3\\
		&= 27M^3-27M^2+9M-1\\
		&= 9(3M^3-9M^2+M)-1
	\end{align*}
\end{namedframe}
\begin{namedframe}{Case 3}
	Case 3 ($N$ is one greater than a number divisible by 3, and can be expressed as $3M+1$):
	\pause
	\begin{align*}
		N^3 &= (3M+1)^3\\
		&= 27M^3+27M^2+9M+1\\
		&= 9(3M^3+9M^2+M)+1
	\end{align*}
\end{namedframe}
\begin{namedframe}{Conclusion}
	The three cases are \textit{exhaustive} (cover all possibilities), and the result fits the conjecture. Therefore, the statement is true.
\end{namedframe}