\documentclass[mathserif]{beamer}

\usepackage{xparse}
\usepackage{amsmath}
\usepackage{amssymb}
\usepackage{graphicx}

\frenchspacing

\logo{\includegraphics[width=0.07\textwidth]{../Logo}}

\usetheme{Rochester}
\usecolortheme{whale}

\setbeamertemplate{frametitle continuation}[from second][\hfill\insertcontinuationtext]

\AtBeginSection[] {%
	\begin{frame}
		\frametitle{Table of Contents}
		\tableofcontents[currentsection]
	\end{frame}
}

\newenvironment{compactmath}[1][\normalsize]%
	{\begin{minipage}{\textwidth}\vspace{-0.5\baselineskip}#1\begin{equation*}}
	{\end{equation*}\end{minipage}}

\newenvironment{sizedmath}[1]%
	{\begingroup#1\begin{equation*}}
	{\end{equation*}\endgroup}

\newenvironment{namedframe}[1]%
	{\begin{frame}\frametitle{#1}\framesubtitle{\secname}}
	{\end{frame}}

\newenvironment{namedbreakframe}[1]%
	{\begin{frame}[allowframebreaks]\frametitle{#1}\framesubtitle{\secname}}
	{\end{frame}}

\newcommand{\sectionstart}[2]{\begin{frame}\frametitle{#1}\centering\Huge\secname\\\large#2\end{frame}}

\newcommand{\sep}{\pause\vspace{1ex}}
\newcommand{\varsep}[1]{\vspace{#1}}

\newcommand{\vertspace}{\vspace{0.5\baselineskip}}

\DeclareTextFontCommand{\emph}{\bfseries}


\title{Sequences and Series}
\subtitle{Your soon-to-be new best friend $\heartsuit$}
\author{Samantha Unger}
\date{\includegraphics{../LicenseLogo}\\\copyright{} Samantha Unger, 2017}

\DeclareTextFontCommand{\emph}{\bfseries}

\begin{document}
	\frame{\titlepage}
	\section{Summations}
	\begin{namedframe}{Introduction to Sigma notation}
		What is Sigma notation?
		\begin{itemize}[<+(1)->]
			\item Fun, easy-to-use way to sum things up!
			\item Super similar to a counted loop in computer programming
		\end{itemize}
		\pause
		Fun fact: Sigma is the 18th letter of the Greek alphabet, and is transliterated as ``s''.
	\end{namedframe}
	\begin{namedframe}{Using Sigma notation}
		\begin{compactmath}[\Large]
			\sum_{n=1}^{4}n
		\end{compactmath}
		The Sigma notation consists of several components. In the example above, we have:
		\begin{description}
			\item[$n$ under Sigma] Index of summation. Some people use $i$, $k$, or $x$
			\item[$1$] First value of $n$ (can be anything)
			\item[$4$] Term we end on
			\item[$n$ after Sigma] Formula for each turn
		\end{description}
		\sep
		\[\sum_{n=1}^{4}n = 1 + 2 + 3 + 4 = 10\]
	\end{namedframe}
	\begin{namedframe}{Example}
		Let's try some quick maths!

		We'll do this one together.

		There's more on your worksheet.
	\end{namedframe}
	\begin{namedframe}{Some \emph{super-cool} and \emph{super-useful} properties}
		Let's talk about why these work:

		\sep

		Multiplying by a constant:
		\[\sum_{k=m}^{n}ca_k = c\sum_{k=m}^{n}a_k\]

		\sep

		Adding/subtracting:
		\[\sum_{k=m}^{n}(a_k + b_k) = \sum_{k=m}^{n}a_k + \sum_{k=m}^{n}b_k\]
	\end{namedframe}
	\begin{namedbreakframe}{Summation shortcuts ;)}
		\begin{block}{Summing $1$ equals $n$}
			\[\sum_{k=1}^{n}1 = n\]
		\end{block}
		\begin{block}{Summing the constant $c$ equals $c \times n$}
			\[\sum_{k=1}^{n}c = nc\]
		\end{block}
		\framebreak
		\begin{alertblock}{A shortcut when summing $k$ (we will develop this later)}
			\[\sum_{k=1}^{n}k = \frac{n(n+1)}{2}\]
		\end{alertblock}
		\framebreak
		These ones are pretty cool and useful:
		\begin{block}{A shortcut when summing $k^2$}
			\[\sum_{k=1}^{n}k^2 = \frac{n(n+1)(2n+1)}{6}\]
		\end{block}
		\begin{block}{A shortcut when summing $k^3$}
			\[\sum_{k=1}^{n}k^3 = \left(\frac{n(n+1)}{2}\right)^2\]
		\end{block}
	\end{namedbreakframe}
	\begin{namedframe}{Try this!}
		\Huge
		Try the super cool practice word problem on your sheet!
	\end{namedframe}
	\section{Arithmetic Sequences}
	\begin{namedframe}{What is an arithmetic sequence?}
		\begin{itemize}
			\item A sequence of numbers where there is a constant difference between successive terms
			\item Example: $3,5,7,9,11$
		\end{itemize}
		\pause
		We can define the $n\textsuperscript{th}$ term with either of the following formulas:
		\[a_n = a_1 + (n-1)d \qquad a_n = a_m + (n-m)d\]
		Where we have:
		\begin{description}
			\item[$a_n$] the $n\textsuperscript{th}$ term
			\item[$n$] the term number
			\item[$d$] the constant difference
			\item[$m$] the $m\textsuperscript{th}$ term.
			\item[$a_1$] the first term of the series if we start counting from 1
		\end{description}
	\end{namedframe}
	\begin{namedframe}{Sum of finite arithmetic series}
		How can we easily sum a finite arithmetic series?
		\begin{itemize}[<+(1)->]
			\item Let me tell you about my main man Gauss\dots
			\item Pair up your values and divide by 2!
		\end{itemize}
		\pause
		\[S_n = \frac{n}{2}(a_1 + a_n)\]
	\end{namedframe}
	\section{Geometric Sequences}
	\begin{namedframe}{What is a geometric sequence?}
		\begin{itemize}
			\item Follows a pattern where each term in found by multiplying the previous term by a constant called the common ratio
			\item Examples: $3,6,12,24,48$ \hspace{2em} or \hspace{2em} $\frac{1}{2},\frac{1}{4},\frac{1}{8},\frac{1}{16}$
		\end{itemize}
		\pause
		We can define the $n\textsuperscript{th}$ term with any of the following formulas:
		\[a_n = a_{n-1} \times r \qquad a_n = a_1 \times r^{n-1}\]
		Where we have:
		\begin{description}
			\item[$a_n$] the $n\textsuperscript{th}$ term
			\item[$a_{n-1}$] the $(n-1)\textsuperscript{th}$ (previous) term
			\item[$a_1$] the first term of the series
			\item[$n$] the term number
			\item[$r$] the common ratio
		\end{description}
	\end{namedframe}
	\begin{namedframe}{Sum of finite geometric series}
		\begin{equation*}
			\begin{array}{lrrrrrrrrrrrrr}
				S_n  &= &a &+ &ar &+ &ar^2 &+ &ar^3 &+ \dots &+ &ar^{n-1} &  &    \\
				rS_n &= &  &+ &ar &+ &ar^2 &+ &ar^3 &+ \dots &+ &ar^{n-1} &+ &ar^n
			\end{array}
		\end{equation*}
		\begin{align*}
			S_n - rS_n &= a - ar^n\\
			S_n(1 - r) &= a(1-r^n)\\
			\therefore S_n &= a\left(\frac{1-r^n}{1-r}\right)
		\end{align*}
	\end{namedframe}
	\section{The Mind-blowing Part}
	\begin{namedframe}{What if\dots}
		What if I told you that the sum of some \emph{infinite} series wasn't infinite?

		\sep

		What if I told you that we can solve this? Can you sense the excitement?

		\begin{sizedmath}{\Huge}
			\sum_{k=1}^{\infty} 3\left(\frac{1}{2}\right)^{k-1}
		\end{sizedmath}
	\end{namedframe}
\end{document}
