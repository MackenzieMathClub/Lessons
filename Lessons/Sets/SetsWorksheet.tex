\documentclass[letterpaper,12pt]{article}

\usepackage[margin=1in]{geometry}
\usepackage{amsmath}
\usepackage{parskip}
\usepackage{tasks}
\usepackage{fancyhdr}

\frenchspacing

\title{Sets}
\author{Mackenzie Math Club}
\date{November 6, 2017}

\newcommand{\such}{\ |\ }

\pagestyle{fancy}
\fancyhead{}
\fancyfoot{}
\renewcommand{\headrulewidth}{0pt}
\renewcommand{\footrulewidth}{0.4pt}
\cfoot{Page \thepage}
\lfoot{\texttt{mackenziemathclub.github.io}}
\rfoot{\copyright{} Vincent Macri, 2017}

\begin{document}
	\maketitle
	\thispagestyle{fancy}
	\section{Overview}
		\subsection{Notation overview}
			\begin{itemize}
				\item A set is an unordered collection of distinct elements.
				\item A set $A$ is a subset of a set $B$ if all elements in $A$ are in $B$. This also makes $B$ a superset of $A$.
				\item The cardinality of a set $A$, written as $n(A)$, is the number of elements in $A$.
			\end{itemize}
		\subsection{Operations overview}
			For two sets $A$ and $B$, the following operations are defined:
			\begin{description}
				\item[Union] $A \cup B$ is a set containing all elements in either $A$ or $B$.
				\item[Intersection] $A \cap B$ is a set containing all elements in both $A$ \emph{and} $B$.
				\item[Relative complement] $B \setminus A$ is the set of all elements in $B$ that are not in $A$.
				\item[Cartesian product] $A \times B = \{(a,b) \such a \in A \text{ and } b \in B\}$
			\end{description}
	\section{Questions}
		If $A = \{7,3,9,18,24\}$ and $B = \{24,119,9,3\}$, then find:
		\begin{tasks}[style=enumerate](2)
			\task $A \cup B$
			\task $n(A \cup B)$
			\task $A \cap B$
			\task $n(A \cap B)$
			\task $A \setminus B$
			\task $n(A \setminus B)$
			\task $A \times B$
			\task $n(A \times B)$
		\end{tasks}
\end{document}
