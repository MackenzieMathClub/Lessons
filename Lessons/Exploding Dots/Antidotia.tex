\section{Antidotia}
\begin{namedframe}{Subtraction}
	\begin{alertblock}{Theorem 1}
		Subtraction does not exist.
	\end{alertblock}
	\begin{alertblock}{Theorem 2}
		What we call subtraction is just the addition of negative numbers.

		Or, subtraction is the addition of the opposite.
	\end{alertblock}
\end{namedframe}
\begin{namedframe}{The antidot}
	The opposite of a dot is an antidot. I'll call these tods.

	This is one tod in one of our machines:
	\begin{explodingdots}{1}
		\draw (1,1) node[tod]{};
	\end{explodingdots}
\end{namedframe}
\begin{namedframe}{How do tods behave?}
	\only<1>{
		\begin{explodingdots}{1}
			\draw (1,1.5) node[dot]{};
			\draw (1,0.5) node[tod]{};
		\end{explodingdots}
	}
	\only<2>{
		\begin{explodingdots}{1}
			\draw (1,1) node[explode]{Poof!};
		\end{explodingdots}
	}
	\only<3>{
		\begin{explodingdots}{1}
		\end{explodingdots}
	}
\end{namedframe}
\begin{namedframe}{Examples}
	What is $564 - 123$?
	\pause
	\begin{equation*}
		\begin{array}{rrrr}
			  & 5 & 6 & 4\\
			- & 1 & 2 & 3\\\hline
			  & 4 & 4 & 1
		\end{array}
	\end{equation*}
	\pause
	What is $441 - 254$?
	\pause
	\begin{equation*}
		\begin{array}{rrrr}
			  & 4 & 4 & 1\\
			- & 2 & 5 & 4\\\hline
			  & 2 & -1 & -3
		\end{array}
	\end{equation*}
	\pause
	Let's fix this together on the board for society's sake. We'll use the exploding dots method.
\end{namedframe}
\begin{namedframe}{Another method}
	\begin{explodingdots}{3}
		\draw (0.5,1) node[dot]{};
		\draw (1.5,1) node[dot]{};

		\draw (3,1) node[tod]{};

		\draw (5,1.5) node[tod]{};
		\draw (5,1) node[tod]{};
		\draw (5,0.5) node[tod]{};
	\end{explodingdots}
	\pause
	There is another way to fix this which is helpful for doing math mentally.

	\pause
	Let's look the place values:
	\[200 + -10 + -3\]

	\pause
	This is very easy to do mentally.

	\[200 + -10 = 190 + -3\]
	\[190 + -3 = 187\]
\end{namedframe}
